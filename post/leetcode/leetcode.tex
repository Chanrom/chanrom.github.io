% Autogenerated translation of leetcode.md by Texpad
% To stop this file being overwritten during the typeset process, please move or remove this header

\documentclass[12pt]{book}
\usepackage{graphicx}
\usepackage[utf8]{inputenc}
\usepackage[a4paper,left=.5in,right=.5in,top=.3in,bottom=0.3in]{geometry}
\setlength\parindent{0pt}
\setlength{\parskip}{\baselineskip}
\renewcommand*\familydefault{\sfdefault}
\usepackage{hyperref}
\pagestyle{plain}
\begin{document}
\Large

\hrule
date: 2018-04-28
title: "Leetcode总结"
tags:
    - leetcode
author:     "Chanrom"
catalog:    true

\section*{comment: true}

\chapter*{Leetcode}

一直知道自己的算法能力偏弱,但是却没有时间针对性地练习过相关的能力。趁着自己有想找一份暑期实习的计划下开始刷Leetcode,刚开始时,即使是easy的题目也解的比较费力,在刷到100题的时候,解easy题的速度明显快了很多。但是在面对medium难度的题依然要花很长的时间才可以AC。本着融会贯通、不停进步的思想,我将在这个博客记录、总结一下遇到的题目,以便自己以及有兴趣一起学习的同学参考。

目前网络上的Leetcode解题思路不计其数,我们可以随手搜到。但是我想只有在自己脑中形成一个知识体系,才可能在所遇到的新问题得到突破。因此这篇博客会尽量地分门别类地总结类似的题目。

\chapter*{二叉树的遍历}

二叉树的遍历需要熟悉:前序遍历、中序遍历、后序遍历以及层次遍历,包括递归的实现和循环的实现。以下分别给出各个算法的Python实现。

\section*{前序遍历-递归}

递归的实现很简单:
\texttt{python
    def preorderTraversal(self, root):
        res = []
        def pre(root):
            if not root:
                return 
            res.append(root.val)
            pre(root.left)
            pre(root.right)
        pre(root)
        return res
}

\section*{前序遍历-循环}

相比递归实现,循环实现具有更高的效率,其实现需要借助栈来实现:
\texttt{python
    def preorderTraversal(self, root):
        stack = []
        res = []
        while (root or len(stack) != 0): \# 注意这个终止条件
            if root:
                res.append(root.val)
                stack.append(root)
                root = root.left
            else:
                root = stack[len(stack) - 1]
                stack.pop()
                root = root.right
        return res
}

\section*{中序遍历-递归}

基本同上:
\texttt{python
    def inorderTraversal(self, root):
        res = []
        def pre(root):
            if not root:
                return 
            pre(root.left)
            res.append(root.val)
            pre(root.right)
        pre(root)
        return res
}

\section*{中序遍历-循环}

中序遍历的循环实现和前序遍历很类似,无非就是(出栈)打印的时机不一样。在代码里的体现就是\texttt{res.append(root.val)}:
\texttt{python
    def inorderTraversal(self, root):
        stack = []
        res = []
        while (root or len(stack) != 0): \# 注意这个终止条件
            if root:
                stack.append(root)
                root = root.left
            else:
                root = stack[len(stack) - 1]
                res.append(root.val)
                stack.pop()
                root = root.right
        return res
}

\section*{后序遍历-递归}

递归很简单,直接给出代码:
\texttt{python
    def postorderTraversal(self, root):
        res = []
        def pre(root):
            if not root:
                return 
            pre(root.left)
            pre(root.right)
            res.append(root.val)
        pre(root)
        return res
}

\section*{后序遍历-循环}

相比前两个而言,后序遍历会比较复杂一点,因为在后序遍历中,必须最后打印根节点的值,也就是说必须先打印左子树和右子树的值。在具体的实现当中,我们可以使用一个\texttt{last}指针,表示在上一时刻访问多的节点。当某个节点没有左右子节点或者右子节点已经访问过了(此时左子节点肯定也访问过了),这个节点的值才可以被打印出来。以下代码:
\texttt{python
    def postorderTraversal(self, root):
        stack = []
        res = []
        last = None
        while (root or len(stack) != 0): \# 注意这个终止条件
            while root: \# 一直往左下搜索,使节点入栈
                stack.append(root)
                root = root.left
            root = stack[len(stack) - 1] \# 开始分析栈内节点
            if (root.right and root.right != last): \# 如果当前节点有右子节点并且没被访问过,又要开始对右子树的访问
                root = root.right
            else: \# 不是以上情况的时候就可以输出当前根节点了
                res.append(root.val)
                stack.pop()
                last = root \# 记录
                root = None \# 保证下一次处理栈顶元素
        return res
}

\section*{层序遍历-循环}

第一种层序遍历要求的打印顺序是:根节点先打印叶子节点最后打印的。思路是每遍历完当前层所有的节点的时候,将当前层内节点拥有的子节点装入列表,以备下一步再开始之前的步骤:
\texttt{python
    def levelOrder(self, root):
        if not root:
            return []
        ans, level = [], [root]
        while level:
            ans.append([node.val for node in level])
            temp = []
            for node in level:
                temp.extend([node.left, node.right])
            level = [leaf for leaf in temp if leaf]
        return ans
}

\section*{层序遍历2-循环}

第一种层序遍历要求的打印顺序是和之前相反,但这个在Python中不是大问题,直接一条命令就可以\texttt{res[::-1]}

\chapter*{字符串翻转}

字符串的翻转涉及到比较多的题目,比如,从最简单的字符串翻转的概念题到涉及回文串的题。总结一下这类题目对“启后”——在将来的面试中解决新的题目——还是很有帮助的。这里介绍几个这样的题目。

\section*{翻转字符串}

把一个字符串最左侧的若干字符截取下来,保持其顺序,放到原字符串截断后留下的子串前面,这个称之为翻转字符串。举个例子,如果\texttt{A="abcde"},那么\texttt{B="deabc"}是\texttt{A}的一个翻转。如果判断一个字符串是不是另外一字符串的翻转呢?解题的思路就是,如果\texttt{B}是\texttt{A}的一个翻转字符串的话,\texttt{A}必须要在\texttt{(B + B)[1:-1]}中能找到,去掉最后和最前的字符是为了保证\texttt{A="aaaa"},\texttt{B="aa"}的情况是不符合要求的。代码如下:
\texttt{python
    def rotateString(self, A, B):
        if len(A) == 0 and len(B) == 0:
            return True
        if len(A) == 0:
            return False
        s = (B + B)[1:-1]
        if A not in s:
            return False
        else:
            return True
}

\chapter*{排列组合}

我这里的排列组合其实指的很宽泛,只要是涉及到计算若干个字符的排列组合、有效排列组合等等都可以归为这一类。把这些题目放到一起,以便总结。

\section*{括号生成}

给出 n 代表生成括号的对数,请你写出一个函数,使其能够生成所有可能的并且有效的括号组合。例如,给出 n = 3,生成结果为:
\texttt{
[
  "((()))",
  "(()())",
  "(())()",
  "()(())",
  "()()()"
]
}
我自己想到的方法是通过变换字符串中的\texttt{)(}这种组合。举例来说,串\texttt{()()()}的第一个\texttt{)(}变换成\texttt{()}则可以变成一个有效的括号串。
\texttt{python
    def generateParenthesis(self, N):
        valid\_str = set(['()'*n])
        stack = ['()'*n]
        while len(stack) != 0:
            top = stack[len(stack) - 1]
            stack.pop()
            top = [c for c in top]
            for i in xrange(1, len(top) - 1):
                if top[i] == ')' and top[i + 1] == '(':
                    tmp = top[:]
                    tmp[i], tmp[i + 1] = tmp[i + 1], tmp[i]
                    if ''.join(tmp) not in valid\_str:  
                        valid\_str.add(''.join(tmp))
                        stack.append(''.join(tmp))
        return list(valid\_str)
}
网上很多解法,通常是利用递归进行广度搜索:
\texttt{python
    def generateParenthesis(self, N):
        ans = []
        def backtrack(S = '', left = 0, right = 0):
            if len(S) == 2 * N:
                ans.append(S)
                return
            if left $<$ N:
                backtrack(S+'(', left+1, right)
            if right $<$ left:
                backtrack(S+')', left, right+1)
        backtrack()
        return ans
}

\chapter*{两个指针}

使用两个指针可以解决不少的问题,特别是有关数组的题目上,这里从易到难介绍几道题目(从二分搜索到复杂一点的二分搜索33,34)。

\chapter*{回溯}

\end{document}
